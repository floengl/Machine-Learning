\documentclass[a4paper,12pt]{article}

% Packages
\usepackage[utf8]{inputenc}
\usepackage{amsmath}
\usepackage{graphicx}
\usepackage{hyperref}
\usepackage{geometry}

\geometry{a4paper, margin=1in}

% Title and Author
\title{Your Document Title}
\author{Your Name}

\begin{document}

% Title Page
\maketitle

% Abstract
\begin{abstract}
This is the abstract of the document. It provides a brief summary of the content.
\end{abstract}

% Table of Contents
\tableofcontents
\newpage

% Sections
\section{Introduction}
This is the introduction section. Here you can introduce the topic of your document.

\section{Methodology}
This section describes the methodology used in your work.

\subsection{Subsection Example}
This is an example of a subsection.

\section{Preprocessing}
We will do four different measures to properly preprocess the data. These are: data imputation, encoding of categorical variables,
scaling of numerical values and dimensionality reduction. In this section we will look at each dataset and 
describe which tasks can be used on the dataset and how a specific task is performed. Then we will compare the different
settings for all the models and determine their effectiveness using measures described in the previous section. 
In order to make this a meaningful comparison, we will use the default hyperparameters for all classifiers and 
we will split the data equally in the cross-validation.\\
Data leakage occurs when data is preprocessed before it was split into the training and test set. This way information 
from the test set is used to preprocess the training set. For example the MinMaxScaler uses the minimun and maximum values 
to scale these values properly. The values should only be computed on the training set and then be used on both training and 
test set. This can get rather difficult when using cross-validation. In order to avoid this, we will use the pipelines.    
Pipelines are a feature of scikit-learn that allow for the chaining of multiple transformers and estimators and also avoid 
data leakage. We will define a different pipeline for each preprocessing task and variant and chain these pipelines to compare 
the different settings.\\
\textbf{Congressional Voting:} As this dataset contains missing values we will need a strategy to impute these values. 
It is important to note that the missing values are all labeled as \textit{'unknown'}.
As all the values are categorical there are two possible strategies for data imputation.
We can either use the most frequent value of the respective attribute in the training dataset 
or we can treat the missing values as a separate category.



\section{Discussion}
This section discusses the implications of your results.

\section{Conclusion}
This is the conclusion section. Summarize your findings and suggest future work.

% References
\begin{thebibliography}{9}
\bibitem{example1}
Author, \textit{Title of the Book}, Publisher, Year.

\bibitem{example2}
Author, \textit{Title of the Article}, Journal, Volume, Page numbers, Year.
\end{thebibliography}

\end{document}